
\section{INTRODUCTION}

For a musculoskeletal system with $n$ muscles and $m$ output dimensions, the set of all feasible activations for a given static task \textbf{b} is represented here as
$A\textbf{x} \leq b$, where $A \in \mathbb{R}^{m \times n}$ represents the set of linear constraints upon the system, $x \in \mathbb{R}^n$ represents the set of activations capable of meeting those requirements.

Optimal control-need to know what the space looks like
Synergies -we want to differentiate task relevant and task irrelevant variation.




define exploration exploitation
define bayesian priors\posterior distributions
sensory feedback in reducing tasl-irrelevant variation
the manifold
what is spatiotemporal
Temporal constraints on the set of feasible activations help us remove activation sequences which are impossible, given the knowsge we have about miscles.
A muscle's speed can be measured in fiber lengths per second, and there the maximum rate at which a muscle can change in its activation is """"6"" percent per second, meanin that fullmuscle activation from baseline would take n seconds.
Notably, experimental observationsof muscle activation change is different, depending on
muscle size?
muscle co-inervation
distance to spinal cord/brain?
frequency of motor pulses tjat tje muscle is most excitable to
ampunt of training
slow and fast twitch muscle involvement
the velocity of the muscle, (is it lengthening or shortening etc)
current muscle length of the muscle
fatigue state of the muscle
alpha and gamma muscle drive
exciteability in the brain and spinal cord
?????

All of these factors appear to limit or increasw the maximum activation change, and its a complicated balance. To represent the myriad of state-based constraints, we use the maximum recorded change in activation from a finger experiment.
That way, we offer the smallest possible constraint, and make an exceptionally conservative model of temporal constraints.
Different muscles are expected to have diferent constraints on their behavior- regardless we used the absolute-fastest activation change from the X muscle, and constrained all other muscles to act at or slower than that maximum.

Spatial constraints
the position of a misciloskeletal system nonlinearly changes the relationships between muscle activations amd endpoint  force productiom.
consider the moment arms- they change in peculiar ways, which would affect control; control must accomodate for changes in the system.
similarly, the maxima' feasible force that a muscle can produce in a given position, is related to the muscle length.




we were intentional in fixing this experiment to a solely static position, so we would can extract the effects garnered by temporalactivatin constraints, without adding complication of motion.
In removing motion, our experiment controls for changes in moment arms, force max capability and other affect of cjange in position and the effects of those parameters.
s


We applied our approach to these things:

1. A schematic model
2. A realistic model from [briantodo cite]

With this, below are the key observations we identified with our research:
\begin{itemize}
\item{item}
\item{item}
\item{item}
\end{itemize}

With respect to the structure of the activation space, we set forth the following key ideas and findings:
\begin{itemize}
\item{item}
\item{item}
\item{item}
\end{itemize}
and most importantly,
\begin{itemize}
\item{item}
\end{itemize}
